\documentclass[UTF8]{ctexart}
% \usepackage{amsfonts}
% \usepackage{amsmath}
% \usepackage{amssymb}
% \usepackage{amsthm}
% \usepackage{booktabs}
\usepackage{courier}
% \usepackage{float}
\usepackage{geometry}
% \usepackage{graphicx}
% \usepackage{hyperref}
% \usepackage{listings}
\geometry{left=2.54cm,right=2.54cm,top=2.18cm,bottom=3.18cm}

\begin{document}

\title{手掌活体检测——开题技术报告}
\author{无43\ \ 马栩杰\ \ 2014011085\\ 无46\ \ 严靖凯\ \ 2014011192\\ 无46\ \ 黄秀峰\ \ 2014011193\\ 无46\ \ 邢成\ \ 2014011167}
\maketitle

\section{实验要求}

	根据序列帧图像中手指的变化特征,判断是否为活体,并搭建实时演示系统。Windows平台演示。

\section{子模块划分}

	根据实验要求,我们的实验分为以下几个子模块:

	\begin{enumerate}
		\item 手掌检测
		\item 手掌分割
		\item 手指检测
		\item 活体检测
		\item Windows下的GUI实现
	\end{enumerate}

	在我们的讨论中,考虑将手掌检测、手掌分割与手指检测相整合,即用统一的方法完成以上检测。这样可以更加充分地利用待检测的信息。

\section{算法与实现}

	\subsection{手掌轮廓检测}

	\subsection{手掌关键点提取}

	\subsection{活体检测}

	% 这个可写可不写吧
	\subsection{Windows GUI}

\end{document}
