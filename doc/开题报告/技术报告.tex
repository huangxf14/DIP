\documentclass[UTF8]{ctexart}
% \usepackage{amsfonts}
% \usepackage{amsmath}
% \usepackage{amssymb}
% \usepackage{amsthm}
% \usepackage{booktabs}
\usepackage{courier}
% \usepackage{float}
\usepackage{geometry}
% \usepackage{graphicx}
% \usepackage{hyperref}
% \usepackage{listings}
\geometry{left=2.54cm,right=2.54cm,top=2.18cm,bottom=3.18cm}

\begin{document}

\title{手掌活体检测——开题技术报告}
\author{无43\ \ 马栩杰\ \ 2014011085\\ 无46\ \ 严靖凯\ \ 2014011192\\ 无46\ \ 黄秀峰\ \ 2014011193\\ 无46\ \ 邢成\ \ 2014011167}
\maketitle

\section{实验要求}

	根据序列帧图像中手指的变化特征,判断是否为活体,并搭建实时演示系统。Windows平台演示。

\section{子模块划分}

	根据实验要求,我们的实验分为以下几个子模块:

	\begin{enumerate}
		\item 手掌检测
		\item 手掌分割
		\item 手指检测
		\item 活体检测
		\item Windows下的GUI实现
	\end{enumerate}

	在我们的讨论中,考虑将手掌检测、手掌分割与手指检测相整合,即用统一的方法完成这些检测。在下面的具体说明中,将按照我们目前的实现计划进行叙述。

\section{算法与实现}

	\subsection{手掌轮廓检测}

    目前我们计划使用非统计学习的方法进行手掌轮廓检测。其主要原因之一在于我们缺乏大量的训练用数据,难以提取出手掌的特征。采用直接识别的方法时,同样可以综合运用各方面的判据(包括肤色、形状等),并可以略去一些复杂的过程,直接完成手掌轮廓的检测。
    
    具体而言,我们计划采用以下流程进行手掌轮廓检测:
		\begin{enumerate}
			\item 基于图像灰度梯度对图像进行分割
			\item 根据颜色判断分割出的区域是否是人体
			\item 根据区域的形状特征判断该部分是否为手掌,其中具体包括:
				\begin{enumerate}
				  \item 对区域进行预处理,使得区域轮廓基本光滑
				  \item 取区域的重心为原点,计算轮廓到重心的距离
				  \item 判断该距离是否符合手掌特征
				\end{enumerate}
			\item 根据轮廓特征对手掌进行初步的对准
		\end{enumerate}

    以上仅仅是我们目前的计划。由于尚未进行实现,还不清楚其具体效果,实际实现时可能会进行相应的修正与调整。

	\subsection{手掌关键点提取}
	
    在获得手掌轮廓后,为之后活体检测做准备,我们需要提前轮廓上的关键点,典型的如手指根部分叉点、指尖、手指轮廓中点,等等。这一部分的具体实现目前计划采用老师提供的关键点提取程序。若在具体实现中发现我们还需要一些额外的功能特性,则将对代码进行相应修改。

	\subsection{活体检测}

    活体检测所需要应对的攻击方式种类繁多,我们在进行了充分的讨论后,决定主要集中于应对图片的攻击。在该功能实现之后,根据剩余的时间多少,再进一步确定是否要拓展至其他攻击方式,比如视频、手掌上贴纸,等等。
    
    我们目前计划使用以下流程进行检测:
		
		\begin{enumerate}
			
			\item 屏幕上显示手掌框,让手掌框中手指发生一定角度的移动,使用者需要在规定时间内按照屏幕要求发生相应移动。
			
			\item 检测各组关键点之前的夹角,与系统所显示的手掌框中的手指间夹角比对,在角度差距小于阈值时,认为通过测试。
			
			\item 使用者需要进行五次测试,通过测试的次数至少达到三次才认为使用者通过验证,即通过活体检测。
			
		\end{enumerate}
		
\end{document}
