\documentclass[UTF8]{ctexart}
% \usepackage{amsfonts}
% \usepackage{amsmath}
% \usepackage{amssymb}
% \usepackage{amsthm}
% \usepackage{booktabs}
\usepackage{courier}
% \usepackage{float}
\usepackage{geometry}
% \usepackage{graphicx}
\usepackage{hyperref}
% \usepackage{listings}
\geometry{left=2.54cm,right=2.54cm,top=2.18cm,bottom=3.18cm}

\begin{document}

\title{手掌活体检测——实验报告}
\author{无43\ \ 马栩杰\ \ 2014011085\\ 无46\ \ 严靖凯\ \ 2014011192\\ 无46\ \ 黄秀峰\ \ 2014011193\\ 无46\ \ 邢成\ \ 2014011167}
\maketitle

\section{背景介绍}

目前随着机器学习、模式识别领域的迅速发展,诸如人脸检测、指纹检测、掌纹检测等基于生物特征的个体识别方式正逐渐取得广泛的应用。与此同时,正如2017年央视“3·15”晚会中所提及的,针对这些特征的活体检测方法在当前越来越称为问题的核心之一。只有当识别系统具备了活体检测的能力,才能避免攻击者通过事先获取的静态信息进行直接攻击,从而大大增强识别系统的鲁棒性。

具体到掌纹时变与手掌活体的检测问题中,我们需要验证检测系统中拍摄到的手掌是否为活体。关于这方面的文章目前很少,因此我们也借鉴了部分人脸活体检测、指纹活体检测中的思路。

\section{系统设计}


\section{模块设计}


\section{实验}

    \subsection{使用流程}

    \subsection{模拟攻击}

\section{系统评价}



\end{document}
